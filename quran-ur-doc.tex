\documentclass[a4paper]{ltxdoc}

\usepackage{holtxdoc}
\usepackage{url}
\usepackage{booktabs}
\usepackage{listings}
\usepackage{tikz}
\hypersetup{%
  plainpages=false,%
  bookmarksnumbered,%
  pdftitle={The quran-ur Package},%
  pdfkeywords={quran-ur, surah, ayah, juz, hizb, ruku, manzil},%
  pdfauthor={Seiied-Mohammad-Javad Razavian},%
  baseurl={http://mirrors.ctan.org/macros/unicodetex/latex/quran-ur/doc/quran-ur-doc.pdf},%
}
\usepackage[all]{quran-ur}
% because of definition of \XeTeX and \XeLaTeX symbols in bidi,
% I should undef these macro that are also defined in holtxdoc package.
\makeatletter
\bidi@undef\XeTeX
\bidi@undef\XeLaTeX
\makeatother
\usepackage{arabxetex}
%\makeatletter
%\bidi@BeforeBeginEnvironment{declcs}{\vspace*{-4mm}}
%\makeatother

%%%%%%%%%%%%%%%%%%%%%%%%%%%%%%%%%%%%%%%%%%%%%

\def\boxcs#1{\leftline{\fbox{\mcs{#1}}}}
\def\mcs#1{\lr{\texttt{\textbackslash{}#1}}}
\def\tt#1{\lr{\texttt{#1}}}
\lstdefinestyle{BashInputStyle}{
  language=bash,
  basicstyle=\footnotesize\sffamily,
  frame=tb,
  columns=fullflexible,
  backgroundcolor=\color{gray!10},
}

% Define box and box title style
\tikzstyle{mybox} = [draw=black, fill=gray!20, very thick,
    rectangle, rounded corners, inner sep=10pt, inner ysep=20pt]
\tikzstyle{fancytitle} =[fill=gray, text=white]

\def\mx#1#2{\mybox{#1}{#2}{.45\textwidth}}
\def\mxf#1#2{\mybox{#1}{#2}{\textwidth}}

\def\mybox#1#2#3{
\begin{tikzpicture}
\node [mybox] (box){%
    \begin{minipage}[t]{#3}
        #2
    \end{minipage}
};
\node[fancytitle, anchor=west, right=10pt] at (box.north west) {\small \cs{#1}};
\node (hole) [anchor=north east, left=5pt ] at (box.north east) { \tikz\fill[very thick,white] (0,0) circle (12pt); };
\node[ ] at (hole.center) {\includegraphics[width=.05\textwidth]{quran.png}};
\end{tikzpicture}%
}

%%%%%%%%%%%%%%%%%%%%%%%%%%%%%%%%%%%%%%%%%%%%%
\title{\includegraphics[scale=.3]{quran.png}\\
The \xpackage{quran-ur} Package\footnote{To one having devoted his life to the Holy Quran}}
\author{Seiied Mohammad Javad Razavian\\\xemail{javadr@gmail.com}}

\date{\quranurdate, Version \quranurversion\footnote{First release: May 9th, 2019}}

\begin{document}
\maketitle

\tableofcontents
%\newpage

\vfill
\section{Introduction}
The \xpackage{quran} package provides several macros for
typesetting the whole or any parts of the Holy Quran based on its popular divisions.
That package also develops
commands for some translations of the Holy Quran including German, Engilsh, and Persian.
Some people asked me to include the other translations into the \xpackage{quran} package,
but because of some technical issues I decided to create new packages for other translations.
One of these variations is \xpackage{quran-ur} including all Urdu translations of the Holy Quran
provided by \url{tanzil.net}, i.e.
\emph{``Abul A'ala Maududi''}, \emph{``Ahmed Raza Khan''},
\emph{``Ahmed Ali''}, \emph{``Fateh Muhammad Jalandhry''},
\emph{``Tahir ul Qadri''}, \emph{``Syed Zeeshan Haider Jawadi''},
\emph{``Muhammad Junagarhi''}, and \emph{``Muhammad Hussain Najafi.''}


\newpage
\section{Loading The Package}
The package will be loaded in the ordinary way
\cs{usepackage[option]\{quran-ur\}}.
After loading the package, it writes some information about itself to the
terminal and to the log file, too. The information is something like:

\begin{lstlisting}[style=BashInputStyle, language=tex, escapechar={|}]
Package: quran-ur |\quranurdate| v|\quranurversion|
Urdu translations extension to the quran package.
\end{lstlisting}


\section{Options of The Package}\label{sec:qurantypesetting}
There are eight options by the names of \xoption{ahmedali}, \xoption{jalandhry}, \xoption{jawadi},
\xoption{junagarhi}, \xoption{kanzuliman}, \xoption{maududi}, \xoption{najafi}, and \xoption{qadri}.
%\marginpar{\ttfamily ahmedali\smallskip}\marginpar{\ttfamily jalandhry\smallskip}
%\marginpar{\ttfamily jawadi}\marginpar{\ttfamily junagarhi\smallskip}
%\marginpar{\ttfamily kanzuliman}\marginpar{\ttfamily maududi\smallskip}
%\marginpar{\ttfamily najafi}\marginpar{\ttfamily qadri\smallskip}
%\marginpar{\ttfamily all\smallskip}
If you pass any combinations of these options, the package will be enabled to typeset these translations in the way that the \xpackage{quran} package does.
There is also \xoption{all} option which loads all Urdu translations provided by the \xpackage{quran-ur} package.
The package loads \xoption{jalandhry} option by default.

This package is completely built on top of the \xpackage{quran} package, therefore you can pass every options
defined in that package to the \xpackage{quran-ur} package.

\section{Differences between \xpackage{quran} and \xpackage{quran-ur}}
At first glance, \xpackage{quran-ur}  offers all functionalities of \xpackage{quran}. Therefore in this case
both are the same except that \xpackage{quran-ur} adds eight other Urdu translations, namely
`ahmedali', `jalandhry', `jawadi', `junagarhi', `kanzuliman', `maududi', `najafi', and `qadri'.
All Urdu translations provided by the package are shown in table~\ref{tab:urtrans}.

\begin{table}[!htbp]
\centering
\begin{tabular}{|c|l|c|}
    \toprule
    order & \multicolumn{1}{c|}{translator} & option's name \\\midrule
    1 & Ahmed Ali & ahmedali \\
    2 & Fateh Muhammad Jalandhry & jalandhry \\
    3 & Syed Zeeshan Haider Jawadi  & jawadi \\
    4 & Muhammad Junagarhi & junagarhi \\
    5 & Ahmed Raza Khan  &  kanzuliman \\
    6 & Abul A'ala Maududi &  maududi \\
    7 & Muhammad Hussain Najafi &  najafi \\
    8 & Tahir ul Qadri  &  qadri \\
    \bottomrule
\end{tabular}
    \caption{Urdu translations used in the package and their option's names}
    \label{tab:urtrans}
\end{table}

\subsection{How to Change the Current Urdu Translation}
    The \xpackage{quran-ur} package uses jalandhry's translation by default.
    The following macro could be used to change the default Urdu translation.
    \begin{declcs}{urSetTrans}\marg{index}
    \end{declcs}
    The \meta{index} could be an integer or a name; both ``order'' and ``option's name''
    shown in Table~\ref{tab:urtrans} are applicable. Both \cs{urSetTrans\{3\}} and \cs{urSetTrans\{jawadi\}}, for example,
    have the same effect.

\subsection{How to Get the Name of Current Urdu Translation}
    \begin{declcs}{urGetTrans}
    \end{declcs}
    The above macro returns the name of current Urdu translation, i.e.  one of
    `ahmedali', `jalandhry', `jawadi', `junagarhi', `kanzuliman', `maududi', `najafi', and `qadri'.

\subsection{How to Typeset the Urdu Translation}
    It's completely similar to the \xpackage{quran} package with extra \texttt{ur} at the end of macro's name.
    All the following macros are usable.

\begin{multicols}{2}
    \begin{itemize}
        \item \cs{quransurahur}
        \item \cs{quranayahur}
        \item \cs{quranpageur}
        \item \cs{quranjuzur}
        \item \cs{quranhizbur}
        \item \cs{quranquarterur}
        \item \cs{quranrukuur}
        \item \cs{quranmanzilur}
        \item \cs{qurantextur}
    \end{itemize}
\end{multicols}

\def\printsurah#1{\centerline{\mxf{urSetTrans\{#1\}\textbackslash{}quransurahur*}
{\setLTR\urSetTrans{#1}\begin{arab}\quransurahur*\end{arab}}}}

    \centerline{\mxf{quransurah*}
    {\begin{arab}\small\quransurah*\end{arab}}}
    \printsurah{ahmedali}
    \printsurah{jalandhry}
    \printsurah{jawadi}
    \printsurah{junagarhi}
    \printsurah{kanzuliman}
    \printsurah{maududi}
    \printsurah{najafi}
    \printsurah{qadri}

\end{document}

